
\chapter{Considerações Finais} \label{chp:consideracoes-finais}

Os objetivos principais deste trabalho  consistiram na definição e realização de
um estudo  computacional do \gls{pma}  conduzido a partir  de um conjunto  de 40
instâncias baseadas em simulação de tráfego e de rede. Foram exploradas técnicas
de  solução  exata utilizando  \gls{pli}  e  Relaxações Lagrangianas  e  métodos
heurísticos capazes de gerar limitantes inferiores e superiores.

Considerando os métodos exatos, foram adaptados pré-processamentos, apresentados
por   Ribeiro   et   al.   \cite{tiago:2019},   para   fixação   de   variáveis.
Desenvolveram-se dois modelos matemáticos,  \gls{dmfm-pma} e \gls{ab-pma}. Ambos
se mostraram eficazes em resolver de maneira  ótima instâncias com até 125 nós e
gerando  soluções viáveis  para todas  as 40  instâncias do  conjunto de  teste.
Ainda, o  \gls{dmfm-pma} serviu de  base para  o desenvolvimento de  um conjunto
contendo quatro Relaxações Lagrangianas (\gls{rl}). Tais relaxações apresentaram
bons resultados com baixo consumo de  tempo, fato que permite a possibilidade de
utilização para instâncias maiores.

Em relação  aos métodos  heurísticos, foi desenvolvida  uma heurística  de busca
local  em arborescências,  aplicada conjuntamente  ao algoritmo  de solução  das
\gls{rl}s, objetivando  a geração  e aperfeiçoamento  de soluções  viáveis. Além
disso, foram  desenvolvidas quatro variações da  meta-heurística \gls{brkga}, de
modo que  três delas  variam a  função de  aptidão do  decodificador e  a última
abordagem extrai a frequência de arcos durante a resolução da \gls{rl} e utiliza
essa informação na geração das chaves aleatórias.

Ao final,  todos os algoritmos  foram comparados através de  testes estatísticos
não  paramétricos. Como  não existem  resultados para  o \gls{pma}  presentes na
literatura,  destaca-se  a necessidade  da  comparação  de resultados  entre  as
diferentes  metodologias  desenvolvidas.  De   tal  modo,  é  possível  validar,
utilizando  múltiplas  abordagens,  a  qualidade  dos  limitantes  inferiores  e
superiores. Ao  considerar o  conjunto de  todas as  metodologias desenvolvidas,
destaca-se  o modelo  \gls{ab-pma}  como  a melhor  abordagem  para obtenção  de
\gls{li}s  e  \gls{ls}s.  As  \gls{rl}s   geram  bons  limitantes  inferiores  e
superiores, não sendo melhores que os modelos para todos os casos, mas o fato de
existirem  versões combinatórias  eficientes,  permite a  solução de  instâncias
maiores.  Por fim,  mesmo  testando  diversas funções  de  aptidão e  detectando
melhoras com  a alteração do processo  de geração de chaves  aleatórias, não foi
possível  gerar \gls{ls}s  com o  \gls{brkga}  que estivessem  acima das  demais
abordagens desenvolvidas.

Como  trabalhos futuros,  pode-se destacar  o possibilidade  de tornar  o modelo
\gls{ab-pma} mais  forte, por  exemplo considerando  um conjunto  exponencial de
restrições para garantir conectividade entre nós.  Por ser de aplicação geral em
arborescências, a  heurística de busca  local desenvolvida pode ser  aplicada em
outras  meta-heurísticas.  Ainda, pode-se  expandir  o  conjunto de  instâncias,
tratando de simulações  cada vez maiores em quantidade de  veículos ou até mesmo
considerando dados que se baseiam em cenários reais.

As técnicas desenvolvidas neste trabalho podem ser aplicadas em outras variantes
do \gls{pma} como \gls{mrp-qos} considerando  as mesmas métricas de \gls{qos} ou
adicionando  mais restrições  ao conjunto,  tais  como o  número de  saltos e  a
estimativa de  duração do enlace. Ainda  é possível gerar resultados  para novas
variantes que  combinam o  \gls{pma} com  o \gls{mrp-qos},  ou seja,  o objetivo
consiste em decidir quais terminais serão  atendidos e ao mesmo tempo computar o
caminho de menor custo no atendimento desse terminal.




% a frequência dos arcos da utilizados
% na relaxação lagrangiana como um viés aplicado às chaves aleatórias geradas.
% ,  de  modo  que  o  primeiro  deles
% (denotado  por \gls{dmfm-pma})  consegue  encontrar soluções  ótimas para  quase
% todas as  instâncias de até 100  nós no tempo máximo  de uma hora e  serviu como
% base  para   o  desenvolvimento   de  um   conjunto  de   diferentes  relaxações
% lagrangianas.  O segundo  modelo  (denotado por  \gls{ab-pma}),  sendo a  última
% abordagem desenvolvida nesta dissertação,  conseguiu gerar soluções viáveis para
% todas as 40 instâncias do conjunto, obtendo dentre elas 28 soluções ótimas. Além
% disso,  ambos modelos  obtiveram o  melhor desempenho  na geração  de limitantes
% inferiores  e o  modelo \gls{ab-pma}  se mostrou  a metodologia  mais eficaz  na
% geração de limitantes superiores.
