\chapter{Revisão Bibliográfica} \label{chp:trab-relacionados}

Neste  Capítulo, descrevemos  as abordagens  e resultados  obtidos por
trabalhos disponíveis na literatura.  Nosso  foco são os trabalhos que
realizaram estudos sobre  algoritmos para o problema  de roteamento em
VANETs e novos protocolos otimizados.

Segundo Peterson \cite{Peterson:2011}, um protocolo fornece um serviço
de comunicação que objetos (nós da  rede) usam para troca de mensagens
e informações.  No caso da  comunicação {\em multicast}, os protocolos
são utilizados  para enviar  informações de uma  fonte para  um grande
conjunto de  destinos utilizando  apenas uma  operação de  envio.  Uma
parte  essencial   dos  protocolos  de  comunicação   é  o  roteamento
\cite{OLIVEIRA20051953}. Do  ponto de vista matemático,  pode-se ver o
roteamento como  um algoritmo  que manipula os  usuários e  enlaces da
rede, de modo  que os pacotes enviados  por um dos nós  pode seguir um
caminho para o destino utilizando  apenas enlaces selecionados e, caso
existam restrições  de \gls{qos},  deve-se garantir  que todos  os nós
destinos recebem as informações  respeitando os limites impostos pelas
métricas  utilizadas na  rede. Assim,  desenvolver procedimentos  para
construção  de   rotas  otimizadas,   significa  otimizar   também  os
protocolos de comunicação.

Para construção  de uma rota  é necessário  primeiramente que o  nó de
origem (raiz) tenha conhecimento da topologia da rede como um todo.  O
processo  de  reconhecimento dos  nós  pode  variar  de acordo  com  o
protocolo, entretanto  o método  mais utilizado é  o de  inundação (do
inglês  \textit{flooding}),  que  consiste  em  um  nó  realizando  um
\textit{broadcast}  para todos  os nós  ao  seu alcance.   Os nós  que
recebem  a mensagem,  por sua  vez, replicam  para os  demais da  rede
utilizando a mesma abordagem. Ao final, o nó que inicialmente enviou a
mensagem consegue ter conhecimento total da topologia da rede. Perceba
que  esse processo  não garante  que um  veículo, após  o processo  de
reconhecimento da rede, se mantenha conectado.

\begin{comment}
% Isso aqui vai para introdução
O  problema de  construção de  rotas  {\em multicast}  é conhecido  na
literatura como \gls{mrp}. Quando  leva-se em consideração as métricas
de \gls{qos}, o problema resultante é conhecido como \gls{mrp-qos}. No
geral, a diferença principal entre  as várias versões do \gls{mrp-qos}
está na  combinação de  métricas de \gls{qos}  utilizadas e  os custos
associados com a  função objetivo.  Existe uma  dificuldade na geração
de instâncias para  o \gls{mrp-qos} associada com o fato  de que, para
garantir a  existência de pelo menos  uma solução viável, o  valor das
métricas  de  \gls{qos}  devem  ser  suficientes  para  que  todos  os
terminais  sejam  atendidos,  mas  não é  vantajoso  utilizar  valores
folgados  a pontos  de  não  haver dificuldade  para  o modelo.   Para
validar a instâncias e identificar o  impacto do valor das métricas de
\gls{qos}  é possível  utilizar  o \gls{pma}.   Uma  solução ótima  do
\gls{pma} determina quais terminais podem  ser atendidos de acordo com
as métricas  de \gls{qos}  informadas, fazendo  com que  seja possível
adaptar a nova instâncias desconsiderando os terminais não atendidos e
computando  o \gls{mrp-qos}  reduzindo  o conjunto  de terminais  para
apenas os atendidos.

Segundo Peterson \cite{Peterson:2011}, um protocolo fornece um serviço
de comunicação que objetos (nós da  rede) usam para troca de mensagens
e informações.   No caso da  comunicação multicast, os  protocolos são
utilizados  para  enviar  informações  de uma  fonte  para  um  grande
conjunto de  destinos utilizando  apenas uma  operação de  envio.  Uma
parte  essencial   dos  protocolos  de  comunicação   é  o  roteamento
\cite{OLIVEIRA20051953}. Do  ponto de vista matemático,  pode-se ver o
roteamento como  um algoritmo  que manipula os  usuários e  enlaces da
rede, de modo  que os pacotes enviados  por um dos nós  pode seguir um
caminho para o destino utilizando  apenas enlaces selecionados e, caso
existam restrições  de \gls{qos},  deve-se garantir  que todos  os nós
destinos recebem as informações  respeitando os limites impostos pelas
métricas  utilizadas na  rede. Assim,  desenvolver procedimentos  para
construção  de   rotas  otimizadas,   significa  otimizar   também  os
protocolos de comunicação.
\end{comment}

Oliveira e Pardalos \cite{OLIVEIRA20051953} afirmam que o \gls{mrp} se
assemelha  com o  problema clássico  da Árvore  Mínima de  Steiner, do
inglês \gls{smt}, o qual é $\mathcal{NP}$-difícil e possui uma extensa
literatura,    por     exemplo    \cite{Garey:1990,    MACULAN1987185,
  Winter:1987}.  A \gls{smt} pode  ser utilizada tanto para formulação
quanto para representação  da solução em aplicações  do \gls{mrp}.  As
versões mais estudadas  do \gls{smt}, associadas ao  \gls{mrp}, são as
que  consideram  atraso  fim   a  fim  \cite{OLIVEIRA20051953}.   Vale
ressaltar ainda que a combinação  de multiplas restrições de \gls{qos}
tornam o problema ainda mais difícil de ser resolvido.

Bitam e Mellouk \cite{BITAM2013981} desenvolveram um algoritmo baseado
em enxame de abelhas para resolver  \gls{mrp-qos} em VANETs. Dado o nó
origem e um conjunto de nós  destino, o objetivo é computar uma árvore
que atenda todos  os terminais na qual a soma  dos custos das conexões
seja mínimo e todas as métricas  de \gls{qos} sejam respeitadas.  As 4
métricas  contidas   em  cada   conexão  são:  custo   de  utilização,
\textit{delay}, \textit{jitter} e largura de banda. O peso atribuído a
cada  métrica na  função  objetivo corresponde  à  sua importância  de
atendimento  para qualidade  de  serviço.  O  algoritmo  de enxame  de
abelhas foi comparado com outros  da literatura em um cenário simulado
contendo  20 nós  (veículos)  em  uma área  de  1200 metros  quadrados
durante um período de 120 segundos  e, por se tratar de uma simulação,
os veículos  trafegam de forma  aleatória no trecho de  mapa fornecido
para o  simulador.  A  criação das  instâncias foi  feita com  base na
obtenção da topologia da rede no tempo de 85 segundos da simulação, ou
seja, é computada uma fotografia que mostra o posicionamento dos nós e
o valor  das métricas para todas  as conexões existentes na  rede.  Os
resultados mostraram que  o algoritmo de enxame  de abelhas apresentou
melhorias em  comparação com demais trabalhos,  incluindo convergência
mais rápida para a melhor solução.

Os  autores Sebastian  {\em et  al.}  \cite{5506245}  apresentaram uma
mudança  no  esquema  de  roteamento  para  fazer  a  disseminação  de
mensagens de  segurança mais eficientemente  em VANETs. O  problema de
roteamento  \textit{multicast} foi  formulado como  uma \gls{smt}  com
restrições de atraso.  O esquema de roteamento detecta os veículos que
estão envolvidos e  ameaçados em casos de acidentes e  os coloca em um
conjunto  de destinatários  das  mensagens de  alerta.   Há também  um
recurso que  visa aumentar a  eficiência dos canais de  comunicação da
rede sem  fio, visto  que há  uma seleção dos  nós que  receberão tais
alertas de emergência, reduzindo assim o número de mensagens enviadas.

Fazio {\em et al.}   \cite{fazio:2013} desenvolveram um novo protocolo
para VANETs  que tem  em seu  núcleo um  algoritmo para  construção de
rotas otimizadas.  Esse algoritmo  leva em consideração três métricas:
\textit{delay} fim  a fim,  interferência co-canal e  probabilidade de
duração da conexão.  A probabilidade  de duração do enlace é calculada
por um método proposto pelos  próprios autores. O objetivo do trabalho
é  encontrar   uma  árvore  geradora  mínima   considerando  todas  as
restrições de  \gls{qos}.  A  validação do protocolo  desenvolvido foi
feita com base na comparação de desempenho, utilizando um simulador de
redes,  com   outros  protocolos  clássicos.   O   protocolo  proposto
apresentou resultados superiores na  avaliação das métricas de largura
de banda, taxa  de entrega de pacotes  e atraso fim a  fim.  Porém, os
autores  notaram uma  pequena  sobrecarga de  processamento.  Além  da
simulação do protocolo,  houve a avaliação do algoritmo  em relação ao
seu desempenho em construir rotas de acordo com as restrições.

Ribeiro {\em et al.} \cite{tiago:2019} desenvolveram uma formulação de
\gls{pli}  e uma  heurística  \textit{relax-and-fix}  para resolver  o
\gls{mrp-qos} aplicado às  VANETs. Dado um nó origem e  um conjunto de
nós destino,  o objetivo é encontrar  a árvore que minimiza  a soma da
quantidade de  saltos e \textit{delay}  fim a fim no  caminho, visando
maximizar a seleção de \textit{links}  com maior estimativa de duração
de  conexão   e  obedecendo  os  limites   máximos  de  \textit{delay}
acumulado,   \textit{jitter}   acumulado,    variação   máxima   entre
\textit{delays}  totais para  todos os  pares de  caminhos e  o limite
mínimo  de  largura de  banda  para  cada  conexão. As  instâncias  de
\textit{benchmark}  foram  geradas  utilizando  simuladores,  buscando
proximidade com configurações reais. O  trabalho não faz comparação da
heurística com a literatura, mas  apresenta um conjunto de técnicas de
pré-processamento  e  a  diferença  de desempenho  relacionada  com  a
utilização   dessas  técnicas   em   conjunto  com   o   modelo  e   o
\textit{relax-and-fix}.

É comum que autores abordem otimizações em redes VANETs propondo novas
extensões  de protocolos  de  comunicação  com algoritmos  otimizados.
Souza  \cite{souza:2012} apresentou  um protocolo  de roteamento  para
VANETs que é baseado na  meta-heurística colônia de formigas. Com base
no protocolo  MAODV ({\em Multicast  Ad hoc On-demand  Distance Vector
  routing  protocol})  \cite{royer:2001},   o  autor  desenvolveu  uma
extensão  que  utiliza informações  de  mobilidade  dos veículos  para
aumentar  a   estabilidade  do  roteamento   \textit{multicast},  essa
extensão é chamada  MAV-AODV. Para construção e  manutenção de árvores
\textit{multicast}  foi utilizado  o algoritmo  baseado em  colônia de
formigas. O método  de avaliação do protocolo utilizou  um ambiente de
simulador de redes com uma quantidade variável de veículos (entre 25 e
100), por 150  segundos e em um cenário de  1600 $\times$ 1500 metros.
Para  análise do  desempenho foram  utilizados dois  outros protocolos
(MAODV  e  PUMA,  {\em   Protocol  for  Unified  Multicasting  through
  Announcement}  \cite{garcia:2004}) e  cinco  métricas de  \gls{qos}:
\textit{delay}  fim  a fim,  variação  do  \textit{delay} fim  a  fim,
sobrecarga de roteamento,  redundância e taxa de  entrega dos pacotes.
Em  relação  ao MAODV,  o  MAV-AODV  obteve resultados  superiores  na
maioria das métricas.  Porém, ao comparar  os resultados com o PUMA, o
protocolo  MAV-AODV  obteve melhor  desempenho  apenas  na métrica  de
redundância de pacotes.

Correia {\em et al.} \cite{correia:2011} desenvolveram um protocolo de
roteamento   baseado  no   DYMO   ({\em   Dynamic  MANET   On-demand})
\cite{gupta:2013} que utiliza uma  estratégia de otimização baseada no
algoritmo de colônia de formigas. Por ser um protocolo reativo, o DYMO
permite a integração do procedimento de otimização efetuando operações
do algoritmo de colônia de formigas,  tais como adição e evaporação de
feromônios nas  rotas disponíveis. A estratégia  de otimização utiliza
também a  velocidade e posição  dos veículos para auxiliar  na escolha
das rotas.  Este protocolo foi validado em um software de simulação de
rede,  com  um  cenário  de  tamanho  1600  $\times$  1500  metros,  a
quantidade de veículos dividida em 5  tamanhos, 25, 50, 75, 100 e 150,
e comparado  com dois outros protocolos,  DYMO nativo e AODV  ({\em Ad
  hoc On-demand Distance  Vector routing protocol}) \cite{royer:2001}.
Para fins  de comparação, fora  consideradas 3 métricas  de \gls{qos}:
taxa  média  de  entrega  dos  pacotes, \textit{delay}  fim  a  fim  e
sobrecarga  de roteamento.   A  análise dos  resultados confirmou  uma
melhora do  novo protocolo  em relação  ao DYMO  nativo apenas  para a
métrica de \textit{delay} fim a fim, enquanto o mesmo protocolo obteve
melhor  desempenho  para  as  métricas  de taxa  média  de  entrega  e
sobrecarga de roteamento em relação ao AODV.

Bitam  {\em  et al.}   \cite{bitam:2013}  propuseram  um protocolo  de
roteamento  híbrido  que  utiliza   as  vantagens  dos  protocolos  de
roteamento  baseados em  topologia em  conjunto com  as vantagens  dos
protocolos de  roteamento baseados  em informações  geográficas.  Este
protocolo utiliza uma abordagem baseada em colônia de abelhas e aplica
o conceito de roteamento \textit{unicast},  ou seja, existe uma origem
e um  destino por vez.   Além disso, o  algoritmo possui um  módulo de
otimização, baseado em algoritmo genético, cuja finalidade é tratar do
roteamento  com base  em informações  geográficas. O  resultado a  ser
retornado por este módulo  consiste em uma rota entre o  nó origem e o
destino, levando  em consideração o  posicionamento dos nós,  custo de
utilização das conexões, \textit{delay} fim a fim e a largura de banda
mínima. Este  protocolo foi validado  utilizando simulador de  redes e
comparando os resultados com os protocolos AODV (baseado em topologia)
e  GPRS  ({\em  General  Packet  Radio  Service})  que  é  baseado  em
informações  geográficas.  A  movimentação  dos veículos  da rede  foi
simulada com base em um modelo de  tráfego em um trecho de 1000 metros
quadrados da cidade de Biskra, na Argélia. Os experimentos trataram de
20,  35  e  50  veículos,   analisando  três  métricas  de  \gls{qos},
\textit{delay} fim  a fim, taxa  de pacotes entregues e  sobrecarga na
rede.   O  protocolo  proposto   apresentou  bom  desempenho,  obtendo
melhores resultados  se tratando  da métrica  de \textit{delay}  fim a
fim.  Enquanto  que para a  métrica de  taxa de pacotes  entregues, os
resultados foram competitivos com  os resultados do AODV, apresentando
uma diferença inferior a 1$\%$. Por  fim, para a métrica de sobrecarga
da rede, o protocolo proposto apresentou vantagem sobre o AODV e perda
em relação ao GPRS.

Saleet {\em  et al.}   \cite{saleet:2011} desenvolveram uma  classe de
protocolos   para  VANETs   chamado   IGRP  ({\em   Intersection-based
  Geographical   Routing   Protocol}),   que   apresentou   resultados
superiores em comparação com demais protocolos de roteamento aplicados
em ambientes  urbanos.  A ideia principal  do IGRP está na  seleção de
interseções de  ruas pelas quais o  pacote deve passar para  atingir o
seu destino.   Esta seleção é feita  de modo que existe  uma garantia,
com  alta  probabilidade,  de  conectividade  da  rede  ao  longo  das
interseções, enquanto satisfaz as  métricas de \gls{qos} utilizadas. O
IGRP faz uso de uma rotina  baseada em algoritmo genético para seleção
de  interseções.  O  protocolo  proposto foi  avaliado utilizando  uma
ferramenta para simulação  de redes que executou por  1000 segundos em
um cenário contendo  entre 150 e 600 nós. As  comparações para análise
de  desempenho  se  deram  entre o  protocolo  desenvolvido  e  demais
trabalhos  contidos na  literatura,  OLSR ({\em  Optimized Link  State
  Routing protocol}) \cite{clausen:2003},  GPSR ({\em Greedy Perimeter
  Stateless Routing})  \cite{karp:2000} e GPCR ({\em  Greedy Perimeter
  Coordinator  Routing})  \cite{lochert:2005}.   O  IGRP  alcançou  os
melhores  resultados  se tratando  das  métricas  de \textit{delay}  e
quantidade de saltos, o que influencia  diretamente na taxa de erro de
entrega  de mensagens,  na  qual o  protocolo  proposto também  obteve
vantagem em relação aos demais.
